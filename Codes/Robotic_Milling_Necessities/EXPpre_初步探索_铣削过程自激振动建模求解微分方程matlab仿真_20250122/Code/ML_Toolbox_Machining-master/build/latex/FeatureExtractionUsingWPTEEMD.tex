%% Generated by Sphinx.
\def\sphinxdocclass{report}
\documentclass[letterpaper,10pt,english]{sphinxmanual}
\ifdefined\pdfpxdimen
   \let\sphinxpxdimen\pdfpxdimen\else\newdimen\sphinxpxdimen
\fi \sphinxpxdimen=.75bp\relax

\PassOptionsToPackage{warn}{textcomp}
\usepackage[utf8]{inputenc}
\ifdefined\DeclareUnicodeCharacter
% support both utf8 and utf8x syntaxes
\edef\sphinxdqmaybe{\ifdefined\DeclareUnicodeCharacterAsOptional\string"\fi}
  \DeclareUnicodeCharacter{\sphinxdqmaybe00A0}{\nobreakspace}
  \DeclareUnicodeCharacter{\sphinxdqmaybe2500}{\sphinxunichar{2500}}
  \DeclareUnicodeCharacter{\sphinxdqmaybe2502}{\sphinxunichar{2502}}
  \DeclareUnicodeCharacter{\sphinxdqmaybe2514}{\sphinxunichar{2514}}
  \DeclareUnicodeCharacter{\sphinxdqmaybe251C}{\sphinxunichar{251C}}
  \DeclareUnicodeCharacter{\sphinxdqmaybe2572}{\textbackslash}
\fi
\usepackage{cmap}
\usepackage[T1]{fontenc}
\usepackage{amsmath,amssymb,amstext}
\usepackage{babel}
\usepackage{times}
\usepackage[Bjarne]{fncychap}
\usepackage{sphinx}

\fvset{fontsize=\small}
\usepackage{geometry}

% Include hyperref last.
\usepackage{hyperref}
% Fix anchor placement for figures with captions.
\usepackage{hypcap}% it must be loaded after hyperref.
% Set up styles of URL: it should be placed after hyperref.
\urlstyle{same}
\addto\captionsenglish{\renewcommand{\contentsname}{Contents}}

\addto\captionsenglish{\renewcommand{\figurename}{Fig.\@ }}
\makeatletter
\def\fnum@figure{\figurename\thefigure{}}
\makeatother
\addto\captionsenglish{\renewcommand{\tablename}{Table }}
\makeatletter
\def\fnum@table{\tablename\thetable{}}
\makeatother
\addto\captionsenglish{\renewcommand{\literalblockname}{Listing}}

\addto\captionsenglish{\renewcommand{\literalblockcontinuedname}{continued from previous page}}
\addto\captionsenglish{\renewcommand{\literalblockcontinuesname}{continues on next page}}
\addto\captionsenglish{\renewcommand{\sphinxnonalphabeticalgroupname}{Non-alphabetical}}
\addto\captionsenglish{\renewcommand{\sphinxsymbolsname}{Symbols}}
\addto\captionsenglish{\renewcommand{\sphinxnumbersname}{Numbers}}

\addto\extrasenglish{\def\pageautorefname{page}}

\setcounter{tocdepth}{3}
\setcounter{secnumdepth}{3}


\title{Feature Extraction Using WPT/EEMD Documentation}
\date{Feb 14, 2020}
\release{0.1}
\author{Melih C. Yesilli, Firas A. Khasawneh, Andreas Otto}
\newcommand{\sphinxlogo}{\vbox{}}
\renewcommand{\releasename}{Release}
\makeindex
\begin{document}

\pagestyle{empty}
\sphinxmaketitle
\pagestyle{plain}
\sphinxtableofcontents
\pagestyle{normal}
\phantomsection\label{\detokenize{index::doc}}


This toolbox includes the documentation for the Python codes that extract features by using Wavelet Packet Transform (WPT)
and Ensemble Empirical Mode Decomposition (EEMD) and diagnose chatter in turning process. Algorithms are based on the methods
explained in \sphinxcite{index:yesilli2019}.


\chapter{Wavelet Packet Transform (WPT)}
\label{\detokenize{WPT:module-WPT_Feature_Extraction}}\label{\detokenize{WPT:wavelet-packet-transform-wpt}}\label{\detokenize{WPT::doc}}\index{WPT\_Feature\_Extraction (module)@\spxentry{WPT\_Feature\_Extraction}\spxextra{module}}

\section{Feature extraction and supervised classification using WPT}
\label{\detokenize{WPT:feature-extraction-and-supervised-classification-using-wpt}}
This function take the time series belong to turning process and
their frequency domain feature matrices computed in Matlab as input. It perfoms 
the 2 class classifications with four different classifiers which
user should specify in parameters section.
\index{WPT\_Feature\_Extraction() (in module WPT\_Feature\_Extraction)@\spxentry{WPT\_Feature\_Extraction()}\spxextra{in module WPT\_Feature\_Extraction}}

\begin{fulllineitems}
\phantomsection\label{\detokenize{WPT:WPT_Feature_Extraction.WPT_Feature_Extraction}}\pysiglinewithargsret{\sphinxcode{\sphinxupquote{WPT\_Feature\_Extraction.}}\sphinxbfcode{\sphinxupquote{WPT\_Feature\_Extraction}}}{\emph{stickout\_length}, \emph{WPT\_Level}, \emph{Classifier}, \emph{plotting}}{}~\begin{quote}\begin{description}
\item[{Parameters}] \leavevmode\begin{itemize}
\item {} 
\sphinxstyleliteralstrong{\sphinxupquote{stickout\_length}} \textendash{} 
The distance between heel of the boring bar and the back surface of the cutting tool
\begin{itemize}
\item {} 
if stickout length is 2 inch, ‘2’

\item {} 
if stickout length is 2.5 inch, ‘2p5’

\item {} 
if stickout length is 3.5 inch, ‘3p5’

\item {} 
if stickout length is 4.5 inch, ‘4p5’

\end{itemize}


\item {} 
\sphinxstyleliteralstrong{\sphinxupquote{WPT\_Level}} \textendash{} Level of Wavelet Packet Decomposition

\item {} 
\sphinxstyleliteralstrong{\sphinxupquote{Classifier}} \textendash{} 
Classifier defined by user
\begin{itemize}
\item {} 
Support Vector Machine: ‘SVC’

\item {} 
Logistic Regression: ‘LR’

\item {} 
Random Forest Classification: ‘RF’

\item {} 
Gradient Boosting: ‘GB’

\end{itemize}


\item {} 
\sphinxstyleliteralstrong{\sphinxupquote{Plotting}} \textendash{} Function will return the plot of the results depending on the number of features used in the classification, if the Plotting is ‘True’.

\end{itemize}

\item[{Returns}] \leavevmode\begin{description}
\item[{results}] \leavevmode
Classification results for training and test set for all combination of ranked features

\item[{time}] \leavevmode
Elapsed time during feature matrix generation and classification

\end{description}

\item[{Example}] \leavevmode\begin{quote}

\begin{sphinxVerbatim}[commandchars=\\\{\}]
\PYG{g+gp}{\PYGZgt{}\PYGZgt{}\PYGZgt{} }\PYG{k+kn}{from} \PYG{n+nn}{WPT\PYGZus{}Feature\PYGZus{}Extraction} \PYG{k+kn}{import} \PYG{n}{WPT\PYGZus{}Feature\PYGZus{}Extraction}
\PYG{g+gp}{\PYGZgt{}\PYGZgt{}\PYGZgt{} }\PYG{k+kn}{import} \PYG{n+nn}{matplotlib.pyplot} \PYG{k+kn}{as} \PYG{n+nn}{plt}
\PYG{g+gp}{\PYGZgt{}\PYGZgt{}\PYGZgt{} }\PYG{k+kn}{from} \PYG{n+nn}{matplotlib} \PYG{k+kn}{import} \PYG{n}{rc}
\PYG{g+gp}{\PYGZgt{}\PYGZgt{}\PYGZgt{} }\PYG{k+kn}{import} \PYG{n+nn}{matplotlib}
\PYG{g+gp}{\PYGZgt{}\PYGZgt{}\PYGZgt{} }\PYG{n}{matplotlib}\PYG{o}{.}\PYG{n}{rcParams}\PYG{o}{.}\PYG{n}{update}\PYG{p}{(}\PYG{p}{\PYGZob{}}\PYG{l+s+s1}{\PYGZsq{}}\PYG{l+s+s1}{font.size}\PYG{l+s+s1}{\PYGZsq{}}\PYG{p}{:} \PYG{l+m+mi}{14}\PYG{p}{\PYGZcb{}}\PYG{p}{)}
\PYG{g+gp}{\PYGZgt{}\PYGZgt{}\PYGZgt{} }\PYG{n}{rc}\PYG{p}{(}\PYG{l+s+s1}{\PYGZsq{}}\PYG{l+s+s1}{font}\PYG{l+s+s1}{\PYGZsq{}}\PYG{p}{,}\PYG{o}{*}\PYG{o}{*}\PYG{p}{\PYGZob{}}\PYG{l+s+s1}{\PYGZsq{}}\PYG{l+s+s1}{family}\PYG{l+s+s1}{\PYGZsq{}}\PYG{p}{:}\PYG{l+s+s1}{\PYGZsq{}}\PYG{l+s+s1}{serif}\PYG{l+s+s1}{\PYGZsq{}}\PYG{p}{,}\PYG{l+s+s1}{\PYGZsq{}}\PYG{l+s+s1}{serif}\PYG{l+s+s1}{\PYGZsq{}}\PYG{p}{:}\PYG{p}{[}\PYG{l+s+s1}{\PYGZsq{}}\PYG{l+s+s1}{Palatino}\PYG{l+s+s1}{\PYGZsq{}}\PYG{p}{]}\PYG{p}{\PYGZcb{}}\PYG{p}{)}
\PYG{g+gp}{\PYGZgt{}\PYGZgt{}\PYGZgt{} }\PYG{n}{rc}\PYG{p}{(}\PYG{l+s+s1}{\PYGZsq{}}\PYG{l+s+s1}{text}\PYG{l+s+s1}{\PYGZsq{}}\PYG{p}{,} \PYG{n}{usetex}\PYG{o}{=}\PYG{n+nb+bp}{True}\PYG{p}{)}
\PYG{g+go}{ }
\PYG{g+go}{\PYGZsh{}parameters}
\PYG{g+go}{ }
\PYG{g+gp}{\PYGZgt{}\PYGZgt{}\PYGZgt{} }\PYG{n}{stickout\PYGZus{}length}\PYG{o}{=}\PYG{l+s+s1}{\PYGZsq{}}\PYG{l+s+s1}{2}\PYG{l+s+s1}{\PYGZsq{}}
\PYG{g+gp}{\PYGZgt{}\PYGZgt{}\PYGZgt{} }\PYG{n}{WPT\PYGZus{}Level} \PYG{o}{=} \PYG{l+m+mi}{4}
\PYG{g+gp}{\PYGZgt{}\PYGZgt{}\PYGZgt{} }\PYG{n}{Classifier} \PYG{o}{=} \PYG{l+s+s1}{\PYGZsq{}}\PYG{l+s+s1}{SVC}\PYG{l+s+s1}{\PYGZsq{}}
\PYG{g+gp}{\PYGZgt{}\PYGZgt{}\PYGZgt{} }\PYG{n}{plotting} \PYG{o}{=} \PYG{n+nb+bp}{True}

\PYG{g+gp}{\PYGZgt{}\PYGZgt{}\PYGZgt{} }\PYG{n}{results} \PYG{o}{=} \PYG{n}{WPT\PYGZus{}Feature\PYGZus{}Extraction}\PYG{p}{(}\PYG{n}{stickout\PYGZus{}length}\PYG{p}{,} \PYG{n}{WPT\PYGZus{}Level}\PYG{p}{,} 
\PYG{g+gp}{\PYGZgt{}\PYGZgt{}\PYGZgt{} }                                 \PYG{n}{Classifier}\PYG{p}{,} \PYG{n}{plotting}\PYG{p}{)}     
\PYG{g+go}{Enter the path of the data files:}
\PYG{g+gp}{\PYGZgt{}\PYGZgt{}\PYGZgt{} }\PYG{n}{D}\PYGZbs{}\PYG{o}{.}\PYG{o}{.}\PYG{o}{.}\PYGZbs{}\PYG{n}{cutting\PYGZus{}tests\PYGZus{}processed}\PYGZbs{}\PYG{n}{data\PYGZus{}2inch\PYGZus{}stickout}
\end{sphinxVerbatim}
\end{quote}

\noindent\sphinxincludegraphics[width=600\sphinxpxdimen,height=360\sphinxpxdimen]{{example}.jpg}

\end{description}\end{quote}

\end{fulllineitems}

\phantomsection\label{\detokenize{WPT:module-WPT_Transfer_Learning}}\index{WPT\_Transfer\_Learning (module)@\spxentry{WPT\_Transfer\_Learning}\spxextra{module}}

\section{Transfer Learning Application Using WPT}
\label{\detokenize{WPT:transfer-learning-application-using-wpt}}
This fuction takes the reconstructed time series after WPT and their freuqency 
domain features as input. The time domain features are computed inside of the 
function. Since this algorithm uses transfer learning principle, user needs to
specify the stickout length of the training set and test set data. The function 
returns classification results in array for both test set and training set.
\index{WPT\_Transfer\_Learning() (in module WPT\_Transfer\_Learning)@\spxentry{WPT\_Transfer\_Learning()}\spxextra{in module WPT\_Transfer\_Learning}}

\begin{fulllineitems}
\phantomsection\label{\detokenize{WPT:WPT_Transfer_Learning.WPT_Transfer_Learning}}\pysiglinewithargsret{\sphinxcode{\sphinxupquote{WPT\_Transfer\_Learning.}}\sphinxbfcode{\sphinxupquote{WPT\_Transfer\_Learning}}}{\emph{stickout\_length\_training}, \emph{stickout\_length\_test}, \emph{WPT\_Level}, \emph{Classifier}}{}~\begin{quote}\begin{description}
\item[{Parameters}] \leavevmode\begin{itemize}
\item {} 
\sphinxstyleliteralstrong{\sphinxupquote{stickout\_length\_training}} \textendash{} 
Stickout length for the training data set
\begin{itemize}
\item {} 
if stickout length is 2 inch, ‘2’

\item {} 
if stickout length is 2.5 inch, ‘2p5’

\item {} 
if stickout length is 3.5 inch, ‘3p5’

\item {} 
if stickout length is 4.5 inch, ‘4p5’

\end{itemize}


\item {} 
\sphinxstyleliteralstrong{\sphinxupquote{stickout\_length\_test}} \textendash{} 
Stickout length for the test data set
\begin{itemize}
\item {} 
if stickout length is 2 inch, ‘2’

\item {} 
if stickout length is 2.5 inch, ‘2p5’

\item {} 
if stickout length is 3.5 inch, ‘3p5’

\item {} 
if stickout length is 4.5 inch, ‘4p5’

\end{itemize}


\item {} 
\sphinxstyleliteralstrong{\sphinxupquote{WPT\_Level}} \textendash{} Level of Wavelet Packet Decomposition

\item {} 
\sphinxstyleliteralstrong{\sphinxupquote{Classifier}} \textendash{} \begin{quote}

Classifier defined by user
\end{quote}
\begin{itemize}
\item {} 
Support Vector Machine: ‘SVC’

\item {} 
Logistic Regression: ‘LR’

\item {} 
Random Forest Classification: ‘RF’

\item {} 
Gradient Boosting: ‘GB’

\end{itemize}


\end{itemize}

\item[{Returns}] \leavevmode\begin{description}
\item[{results}] \leavevmode
Classification results for training and test set for all combination of ranked features

\item[{time}] \leavevmode
Elapsed time during feature matrix generation and classification

\end{description}

\item[{Example}] \leavevmode
\begin{sphinxVerbatim}[commandchars=\\\{\}]
\PYG{g+gp}{\PYGZgt{}\PYGZgt{}\PYGZgt{} }\PYG{k+kn}{from} \PYG{n+nn}{WPT\PYGZus{}Transfer\PYGZus{}Learning} \PYG{k+kn}{import} \PYG{n}{WPT\PYGZus{}Transfer\PYGZus{}Learning}

\PYG{g+go}{\PYGZsh{}parameters}
\PYG{g+go}{ }
\PYG{g+gp}{\PYGZgt{}\PYGZgt{}\PYGZgt{} }\PYG{n}{stickout\PYGZus{}length\PYGZus{}training} \PYG{o}{=} \PYG{l+s+s1}{\PYGZsq{}}\PYG{l+s+s1}{2}\PYG{l+s+s1}{\PYGZsq{}}
\PYG{g+gp}{\PYGZgt{}\PYGZgt{}\PYGZgt{} }\PYG{n}{stickout\PYGZus{}length\PYGZus{}test} \PYG{o}{=} \PYG{l+s+s1}{\PYGZsq{}}\PYG{l+s+s1}{4p5}\PYG{l+s+s1}{\PYGZsq{}}
\PYG{g+gp}{\PYGZgt{}\PYGZgt{}\PYGZgt{} }\PYG{n}{WPT\PYGZus{}Level}\PYG{o}{=}\PYG{l+m+mi}{4}
\PYG{g+gp}{\PYGZgt{}\PYGZgt{}\PYGZgt{} }\PYG{n}{Classifier}\PYG{o}{=}\PYG{l+s+s1}{\PYGZsq{}}\PYG{l+s+s1}{SVC}\PYG{l+s+s1}{\PYGZsq{}}

\PYG{g+gp}{\PYGZgt{}\PYGZgt{}\PYGZgt{} }\PYG{n}{results} \PYG{o}{=} \PYG{n}{WPT\PYGZus{}Transfer\PYGZus{}Learning}\PYG{p}{(}\PYG{n}{stickout\PYGZus{}length\PYGZus{}training}\PYG{p}{,} 
\PYG{g+gp}{\PYGZgt{}\PYGZgt{}\PYGZgt{} }                                    \PYG{n}{stickout\PYGZus{}length\PYGZus{}test}\PYG{p}{,}
\PYG{g+gp}{\PYGZgt{}\PYGZgt{}\PYGZgt{} }                                   \PYG{n}{WPT\PYGZus{}Level}\PYG{p}{,} \PYG{n}{Classifier}\PYG{p}{)}     
\PYG{g+go}{Enter the path of training set data files:}
\PYG{g+gp}{\PYGZgt{}\PYGZgt{}\PYGZgt{} }\PYG{n}{D}\PYGZbs{}\PYG{o}{.}\PYG{o}{.}\PYG{o}{.}\PYGZbs{}\PYG{n}{cutting\PYGZus{}tests\PYGZus{}processed}\PYGZbs{}\PYG{n}{data\PYGZus{}2inch\PYGZus{}stickout}
\PYG{g+go}{Enter the path of test set data files:}
\PYG{g+gp}{\PYGZgt{}\PYGZgt{}\PYGZgt{} }\PYG{n}{D}\PYGZbs{}\PYG{o}{.}\PYG{o}{.}\PYG{o}{.}\PYGZbs{}\PYG{n}{cutting\PYGZus{}tests\PYGZus{}processed}\PYGZbs{}\PYG{n}{data\PYGZus{}4p5inch\PYGZus{}stickout}
\end{sphinxVerbatim}

\end{description}\end{quote}

\end{fulllineitems}

\phantomsection\label{\detokenize{WPT:module-WPT_Transfer_Learning_2case}}\index{WPT\_Transfer\_Learning\_2case (module)@\spxentry{WPT\_Transfer\_Learning\_2case}\spxextra{module}}

\section{Transfer Learning On Two Cases Using WPT}
\label{\detokenize{WPT:transfer-learning-on-two-cases-using-wpt}}
This fuction implement transfer learning by training a classifier on two different 
data sets and testing it on remaining two different data sets. Stickout length of 
each data set should be determined by user.
\index{WPT\_Transfer\_Learning\_2case() (in module WPT\_Transfer\_Learning\_2case)@\spxentry{WPT\_Transfer\_Learning\_2case()}\spxextra{in module WPT\_Transfer\_Learning\_2case}}

\begin{fulllineitems}
\phantomsection\label{\detokenize{WPT:WPT_Transfer_Learning_2case.WPT_Transfer_Learning_2case}}\pysiglinewithargsret{\sphinxcode{\sphinxupquote{WPT\_Transfer\_Learning\_2case.}}\sphinxbfcode{\sphinxupquote{WPT\_Transfer\_Learning\_2case}}}{\emph{stickout\_lengths}, \emph{WPT\_Level}, \emph{Classifier}}{}~\begin{quote}\begin{description}
\item[{Parameters}] \leavevmode\begin{itemize}
\item {} 
\sphinxstyleliteralstrong{\sphinxupquote{stickout\_lengths}} \textendash{} 
Stickout length for the training and test set in a np.array({[}{]}) format.First two stickout length are considered as training set data and the remaining ones are test set data.
\begin{itemize}
\item {} 
if stickout length is 2 inch, ‘2’

\item {} 
if stickout length is 2.5 inch, ‘2p5’

\item {} 
if stickout length is 3.5 inch, ‘3p5’

\item {} 
if stickout length is 4.5 inch, ‘4p5’

\end{itemize}


\item {} 
\sphinxstyleliteralstrong{\sphinxupquote{WPT\_Level}} \textendash{} Level of Wavelet Packet Decomposition

\item {} 
\sphinxstyleliteralstrong{\sphinxupquote{Classifier}} \textendash{} \begin{quote}

Classifier defined by user
\end{quote}
\begin{itemize}
\item {} 
Support Vector Machine: ‘SVC’

\item {} 
Logistic Regression: ‘LR’

\item {} 
Random Forest Classification: ‘RF’

\item {} 
Gradient Boosting: ‘GB’

\end{itemize}


\end{itemize}

\item[{Returns}] \leavevmode\begin{description}
\item[{results}] \leavevmode
Classification results for training and test set for all combination of ranked features

\item[{time}] \leavevmode
Elapsed time during feature matrix generation and classification

\end{description}

\item[{Example}] \leavevmode
\begin{sphinxVerbatim}[commandchars=\\\{\}]
\PYG{g+gp}{\PYGZgt{}\PYGZgt{}\PYGZgt{} }\PYG{k+kn}{from} \PYG{n+nn}{WPT\PYGZus{}Transfer\PYGZus{}Learning\PYGZus{}2case} \PYG{k+kn}{import} \PYG{n}{WPT\PYGZus{}Transfer\PYGZus{}Learning\PYGZus{}2case}

\PYG{g+go}{\PYGZsh{}parameters}
\PYG{g+go}{ }
\PYG{g+gp}{\PYGZgt{}\PYGZgt{}\PYGZgt{} }\PYG{n}{stickout\PYGZus{}lengths} \PYG{o}{=} \PYG{p}{[}\PYG{l+s+s1}{\PYGZsq{}}\PYG{l+s+s1}{2}\PYG{l+s+s1}{\PYGZsq{}}\PYG{p}{,}\PYG{l+s+s1}{\PYGZsq{}}\PYG{l+s+s1}{2p5}\PYG{l+s+s1}{\PYGZsq{}}\PYG{p}{,}\PYG{l+s+s1}{\PYGZsq{}}\PYG{l+s+s1}{3p5}\PYG{l+s+s1}{\PYGZsq{}}\PYG{p}{,}\PYG{l+s+s1}{\PYGZsq{}}\PYG{l+s+s1}{4p5}\PYG{l+s+s1}{\PYGZsq{}}\PYG{p}{]}
\PYG{g+gp}{\PYGZgt{}\PYGZgt{}\PYGZgt{} }\PYG{n}{WPT\PYGZus{}Level}\PYG{o}{=}\PYG{l+m+mi}{4}
\PYG{g+gp}{\PYGZgt{}\PYGZgt{}\PYGZgt{} }\PYG{n}{Classifier}\PYG{o}{=}\PYG{l+s+s1}{\PYGZsq{}}\PYG{l+s+s1}{SVC}\PYG{l+s+s1}{\PYGZsq{}}

\PYG{g+gp}{\PYGZgt{}\PYGZgt{}\PYGZgt{} }\PYG{n}{results} \PYG{o}{=} \PYG{n}{WPT\PYGZus{}Transfer\PYGZus{}Learning\PYGZus{}2case}\PYG{p}{(}\PYG{n}{stickout\PYGZus{}lengths}\PYG{p}{,} 
\PYG{g+gp}{\PYGZgt{}\PYGZgt{}\PYGZgt{} }                                      \PYG{n}{WPT\PYGZus{}Level}\PYG{p}{,} \PYG{n}{Classifier}\PYG{p}{)}     
\PYG{g+go}{Enter the path of first training set data files:}
\PYG{g+gp}{\PYGZgt{}\PYGZgt{}\PYGZgt{} }\PYG{n}{D}\PYGZbs{}\PYG{o}{.}\PYG{o}{.}\PYG{o}{.}\PYGZbs{}\PYG{n}{cutting\PYGZus{}tests\PYGZus{}processed}\PYGZbs{}\PYG{n}{data\PYGZus{}2inch\PYGZus{}stickout}
\PYG{g+go}{Enter the path of second training set data files:}
\PYG{g+gp}{\PYGZgt{}\PYGZgt{}\PYGZgt{} }\PYG{n}{D}\PYGZbs{}\PYG{o}{.}\PYG{o}{.}\PYG{o}{.}\PYGZbs{}\PYG{n}{cutting\PYGZus{}tests\PYGZus{}processed}\PYGZbs{}\PYG{n}{data\PYGZus{}2p5inch\PYGZus{}stickout}
\PYG{g+go}{Enter the path of first test set data files:}
\PYG{g+gp}{\PYGZgt{}\PYGZgt{}\PYGZgt{} }\PYG{n}{D}\PYGZbs{}\PYG{o}{.}\PYG{o}{.}\PYG{o}{.}\PYGZbs{}\PYG{n}{cutting\PYGZus{}tests\PYGZus{}processed}\PYGZbs{}\PYG{n}{data\PYGZus{}3p5inch\PYGZus{}stickout}   
\PYG{g+go}{Enter the path of second test set data files:}
\PYG{g+gp}{\PYGZgt{}\PYGZgt{}\PYGZgt{} }\PYG{n}{D}\PYGZbs{}\PYG{o}{.}\PYG{o}{.}\PYG{o}{.}\PYGZbs{}\PYG{n}{cutting\PYGZus{}tests\PYGZus{}processed}\PYGZbs{}\PYG{n}{data\PYGZus{}4p5inch\PYGZus{}stickout}
\end{sphinxVerbatim}

\end{description}\end{quote}

\end{fulllineitems}



\chapter{Ensemble Empirical Mode Decomposition (EEMD)}
\label{\detokenize{EEMD:module-EEMD_Feature_Extraction}}\label{\detokenize{EEMD:ensemble-empirical-mode-decomposition-eemd}}\label{\detokenize{EEMD::doc}}\index{EEMD\_Feature\_Extraction (module)@\spxentry{EEMD\_Feature\_Extraction}\spxextra{module}}

\section{Feature extraction and supervised classification using EEMD}
\label{\detokenize{EEMD:feature-extraction-and-supervised-classification-using-eemd}}
This function takes time series and decompose them using Ensemble Empirical Mode 
Decomposition (EEMD). If user already computed the decompositions, algorithm will
directly compute the features and generate feature matrices. Algorithm require user 
give classifier name, informative intrinsic mode function (IMF) number and the stickout
length of the data user working on.
\index{EEMD\_Feature\_Extraction() (in module EEMD\_Feature\_Extraction)@\spxentry{EEMD\_Feature\_Extraction()}\spxextra{in module EEMD\_Feature\_Extraction}}

\begin{fulllineitems}
\phantomsection\label{\detokenize{EEMD:EEMD_Feature_Extraction.EEMD_Feature_Extraction}}\pysiglinewithargsret{\sphinxcode{\sphinxupquote{EEMD\_Feature\_Extraction.}}\sphinxbfcode{\sphinxupquote{EEMD\_Feature\_Extraction}}}{\emph{stickout\_length}, \emph{EEMDecs}, \emph{p}, \emph{Classifier}}{}~\begin{quote}\begin{description}
\item[{Parameters}] \leavevmode\begin{itemize}
\item {} 
\sphinxstyleliteralstrong{\sphinxupquote{stickout\_length}} \textendash{} 
The distance between heel of the boring bar and the back surface of the cutting tool
\begin{itemize}
\item {} 
if stickout length is 2 inch, ‘2’

\item {} 
if stickout length is 2.5 inch, ‘2p5’

\item {} 
if stickout length is 3.5 inch, ‘3p5’

\item {} 
if stickout length is 4.5 inch, ‘4p5’

\end{itemize}


\item {} 
\sphinxstyleliteralstrong{\sphinxupquote{EEMDecs}} \textendash{} \begin{itemize}
\item {} 
if decompositions have already been computed, ‘A’

\item {} 
if decompositions have not been computed, ‘NA’

\end{itemize}


\item {} 
\sphinxstyleliteralstrong{\sphinxupquote{p}} \textendash{} Informative intrinsic mode function (IMF) number

\item {} 
\sphinxstyleliteralstrong{\sphinxupquote{Classifier}} \textendash{} 
Classifier defined by user
\begin{itemize}
\item {} 
Support Vector Machine: ‘SVC’

\item {} 
Logistic Regression: ‘LR’

\item {} 
Random Forest Classification: ‘RF’

\item {} 
Gradient Boosting: ‘GB’

\end{itemize}


\end{itemize}

\item[{Returns}] \leavevmode\begin{description}
\item[{results}] \leavevmode
Classification results for training and test set for all combination of ranked features

\item[{time}] \leavevmode
Elapsed time during feature matrix generation and classification

\end{description}

\item[{Example}] \leavevmode
\begin{sphinxVerbatim}[commandchars=\\\{\}]
\PYG{g+gp}{\PYGZgt{}\PYGZgt{}\PYGZgt{} }\PYG{k+kn}{from} \PYG{n+nn}{EEMD\PYGZus{}Feature\PYGZus{}Extraction} \PYG{k+kn}{import} \PYG{n}{EEMD\PYGZus{}Feature\PYGZus{}Extraction}
\PYG{g+go}{ }
\PYG{g+go}{\PYGZsh{}parameters}
\PYG{g+gp}{\PYGZgt{}\PYGZgt{}\PYGZgt{} }\PYG{n}{stickout\PYGZus{}length}\PYG{o}{=}\PYG{l+s+s1}{\PYGZsq{}}\PYG{l+s+s1}{2}\PYG{l+s+s1}{\PYGZsq{}}
\PYG{g+gp}{\PYGZgt{}\PYGZgt{}\PYGZgt{} }\PYG{n}{EEMDecs} \PYG{o}{=} \PYG{l+s+s1}{\PYGZsq{}}\PYG{l+s+s1}{A}\PYG{l+s+s1}{\PYGZsq{}}
\PYG{g+gp}{\PYGZgt{}\PYGZgt{}\PYGZgt{} }\PYG{n}{p}\PYG{o}{=}\PYG{l+m+mi}{2}
\PYG{g+gp}{\PYGZgt{}\PYGZgt{}\PYGZgt{} }\PYG{n}{Classifier} \PYG{o}{=} \PYG{l+s+s1}{\PYGZsq{}}\PYG{l+s+s1}{SVC}\PYG{l+s+s1}{\PYGZsq{}}

\PYG{g+gp}{\PYGZgt{}\PYGZgt{}\PYGZgt{} }\PYG{n}{results} \PYG{o}{=} \PYG{n}{WPT\PYGZus{}Feature\PYGZus{}Extraction}\PYG{p}{(}\PYG{n}{stickout\PYGZus{}length}\PYG{p}{,} \PYG{n}{WPT\PYGZus{}Level}\PYG{p}{,} 
\PYG{g+gp}{\PYGZgt{}\PYGZgt{}\PYGZgt{} }                                 \PYG{n}{Classifier}\PYG{p}{,} \PYG{n}{plotting}\PYG{p}{)}     
\PYG{g+go}{Enter the path of the data files:}
\PYG{g+gp}{\PYGZgt{}\PYGZgt{}\PYGZgt{} }\PYG{n}{D}\PYGZbs{}\PYG{o}{.}\PYG{o}{.}\PYG{o}{.}\PYGZbs{}\PYG{n}{cutting\PYGZus{}tests\PYGZus{}processed}\PYGZbs{}\PYG{n}{data\PYGZus{}2inch\PYGZus{}stickout}
\PYG{g+go}{Enter Enter the path of EEMD files:}
\PYG{g+gp}{\PYGZgt{}\PYGZgt{}\PYGZgt{} }\PYG{n}{D}\PYGZbs{}\PYG{o}{.}\PYG{o}{.}\PYG{o}{.}\PYGZbs{}\PYG{n}{cutting\PYGZus{}tests\PYGZus{}processed}\PYGZbs{}\PYG{n}{data\PYGZus{}2inch\PYGZus{}stickout}
\end{sphinxVerbatim}

\end{description}\end{quote}

\end{fulllineitems}

\phantomsection\label{\detokenize{EEMD:module-EEMD_Transfer_Learning}}\index{EEMD\_Transfer\_Learning (module)@\spxentry{EEMD\_Transfer\_Learning}\spxextra{module}}

\section{Transfer Learning Application Using EEMD}
\label{\detokenize{EEMD:transfer-learning-application-using-eemd}}
This fuction uses tranfer learning principles to diagnose chatter in time series 
obtained from turning cutting experiment. Intrinsic mode functions (IMFs) or decomposition 
for each time series should have been computed to be able to use this code.
\index{EEMD\_Transfer\_Learning() (in module EEMD\_Transfer\_Learning)@\spxentry{EEMD\_Transfer\_Learning()}\spxextra{in module EEMD\_Transfer\_Learning}}

\begin{fulllineitems}
\phantomsection\label{\detokenize{EEMD:EEMD_Transfer_Learning.EEMD_Transfer_Learning}}\pysiglinewithargsret{\sphinxcode{\sphinxupquote{EEMD\_Transfer\_Learning.}}\sphinxbfcode{\sphinxupquote{EEMD\_Transfer\_Learning}}}{\emph{stickout\_length\_training}, \emph{stickout\_length\_test}, \emph{p\_train}, \emph{p\_test}, \emph{Classifier}}{}~\begin{quote}\begin{description}
\item[{Parameters}] \leavevmode\begin{itemize}
\item {} 
\sphinxstyleliteralstrong{\sphinxupquote{stickout\_length\_training}} \textendash{} 
Stickout length for the training data set
\begin{itemize}
\item {} 
if stickout length is 2 inch, ‘2’

\item {} 
if stickout length is 2.5 inch, ‘2p5’

\item {} 
if stickout length is 3.5 inch, ‘3p5’

\item {} 
if stickout length is 4.5 inch, ‘4p5’

\end{itemize}


\item {} 
\sphinxstyleliteralstrong{\sphinxupquote{stickout\_length\_test}} \textendash{} 
Stickout length for the test data set
\begin{itemize}
\item {} 
if stickout length is 2 inch, ‘2’

\item {} 
if stickout length is 2.5 inch, ‘2p5’

\item {} 
if stickout length is 3.5 inch, ‘3p5’

\item {} 
if stickout length is 4.5 inch, ‘4p5’

\end{itemize}


\item {} 
\sphinxstyleliteralstrong{\sphinxupquote{p\_train}} \textendash{} Informative intrinsic mode function (IMF) number for training set

\item {} 
\sphinxstyleliteralstrong{\sphinxupquote{p\_test}} \textendash{} Informative intrinsic mode function (IMF) number for test set

\item {} 
\sphinxstyleliteralstrong{\sphinxupquote{Classifier}} \textendash{} 
Classifier defined by user
\begin{itemize}
\item {} 
Support Vector Machine: ‘SVC’

\item {} 
Logistic Regression: ‘LR’

\item {} 
Random Forest Classification: ‘RF’

\item {} 
Gradient Boosting: ‘GB’

\end{itemize}


\end{itemize}

\item[{Returns}] \leavevmode\begin{description}
\item[{results}] \leavevmode
Classification results for training and test set for all combination of ranked features

\item[{time}] \leavevmode
Elapsed time during feature matrix generation and classification

\end{description}

\item[{Example}] \leavevmode
\begin{sphinxVerbatim}[commandchars=\\\{\}]
\PYG{g+gp}{\PYGZgt{}\PYGZgt{}\PYGZgt{} }\PYG{k+kn}{from} \PYG{n+nn}{EEMD\PYGZus{}Feature\PYGZus{}Extraction} \PYG{k+kn}{import} \PYG{n}{EEMD\PYGZus{}Feature\PYGZus{}Extraction}
\PYG{g+go}{ }
\PYG{g+go}{\PYGZsh{}parameters}
\PYG{g+gp}{\PYGZgt{}\PYGZgt{}\PYGZgt{} }\PYG{n}{stickout\PYGZus{}length\PYGZus{}training}\PYG{o}{=}\PYG{l+s+s1}{\PYGZsq{}}\PYG{l+s+s1}{2}\PYG{l+s+s1}{\PYGZsq{}}
\PYG{g+gp}{\PYGZgt{}\PYGZgt{}\PYGZgt{} }\PYG{n}{stickout\PYGZus{}length\PYGZus{}test}\PYG{o}{=}\PYG{l+s+s1}{\PYGZsq{}}\PYG{l+s+s1}{4p5}\PYG{l+s+s1}{\PYGZsq{}}
\PYG{g+gp}{\PYGZgt{}\PYGZgt{}\PYGZgt{} }\PYG{n}{p\PYGZus{}train} \PYG{o}{=} \PYG{l+m+mi}{2}
\PYG{g+gp}{\PYGZgt{}\PYGZgt{}\PYGZgt{} }\PYG{n}{p\PYGZus{}test} \PYG{o}{=} \PYG{l+m+mi}{1}
\PYG{g+gp}{\PYGZgt{}\PYGZgt{}\PYGZgt{} }\PYG{n}{Classifier} \PYG{o}{=} \PYG{l+s+s1}{\PYGZsq{}}\PYG{l+s+s1}{GB}\PYG{l+s+s1}{\PYGZsq{}}

\PYG{g+gp}{\PYGZgt{}\PYGZgt{}\PYGZgt{} }\PYG{n}{results} \PYG{o}{=} \PYG{n}{EEMD\PYGZus{}Transfer\PYGZus{}Learning}\PYG{p}{(}\PYG{n}{stickout\PYGZus{}length\PYGZus{}training}\PYG{p}{,} 
\PYG{g+go}{                                     stickout\PYGZus{}length\PYGZus{}test, }
\PYG{g+gp}{\PYGZgt{}\PYGZgt{}\PYGZgt{} }                                 \PYG{n}{p\PYGZus{}train}\PYG{p}{,} \PYG{n}{p\PYGZus{}test}\PYG{p}{,}
\PYG{g+go}{                                     Classifier)     }
\PYG{g+go}{Enter the path of training data files:}
\PYG{g+gp}{\PYGZgt{}\PYGZgt{}\PYGZgt{} }\PYG{n}{D}\PYGZbs{}\PYG{o}{.}\PYG{o}{.}\PYG{o}{.}\PYGZbs{}\PYG{n}{cutting\PYGZus{}tests\PYGZus{}processed}\PYGZbs{}\PYG{n}{data\PYGZus{}2inch\PYGZus{}stickout}
\PYG{g+go}{Enter the path of test data files:}
\PYG{g+gp}{\PYGZgt{}\PYGZgt{}\PYGZgt{} }\PYG{n}{D}\PYGZbs{}\PYG{o}{.}\PYG{o}{.}\PYG{o}{.}\PYGZbs{}\PYG{n}{cutting\PYGZus{}tests\PYGZus{}processed}\PYGZbs{}\PYG{n}{data\PYGZus{}4p5inch\PYGZus{}stickout}
\PYG{g+go}{Enter the path to decompositions for training set:}
\PYG{g+gp}{\PYGZgt{}\PYGZgt{}\PYGZgt{} }\PYG{n}{D}\PYGZbs{}\PYG{o}{.}\PYG{o}{.}\PYG{o}{.}\PYGZbs{}\PYG{n}{eIMFs}\PYGZbs{}\PYG{n}{data\PYGZus{}2inch\PYGZus{}stickout}
\PYG{g+go}{Enter the path to decompositions for test set:}
\PYG{g+gp}{\PYGZgt{}\PYGZgt{}\PYGZgt{} }\PYG{n}{D}\PYGZbs{}\PYG{o}{.}\PYG{o}{.}\PYG{o}{.}\PYGZbs{}\PYG{n}{eIMFs}\PYGZbs{}\PYG{n}{data\PYGZus{}4p5inch\PYGZus{}stickout}  
\end{sphinxVerbatim}

\end{description}\end{quote}

\end{fulllineitems}



\chapter{References}
\label{\detokenize{index:references}}

\begin{itemize}
\item {} 
\DUrole{xref,std,std-ref}{genindex}

\item {} 
\DUrole{xref,std,std-ref}{modindex}

\item {} 
\DUrole{xref,std,std-ref}{search}

\end{itemize}

\begin{sphinxthebibliography}{1}
\bibitem[1]{index:yesilli2019}
Melih C. Yesilli, Firas A. Khasawneh, and Andreas Otto. On transfer learning for chatter detection in turning using wavelet packet transform and ensemble empirical mode decomposition. \sphinxstyleemphasis{CIRP Journal of Manufacturing Science and Technology}, dec 2019. \sphinxhref{https://doi.org/10.1016/j.cirpj.2019.11.003}{doi:10.1016/j.cirpj.2019.11.003}.
\end{sphinxthebibliography}


\renewcommand{\indexname}{Python Module Index}
\begin{sphinxtheindex}
\let\bigletter\sphinxstyleindexlettergroup
\bigletter{e}
\item\relax\sphinxstyleindexentry{EEMD\_Feature\_Extraction}\sphinxstyleindexpageref{EEMD:\detokenize{module-EEMD_Feature_Extraction}}
\item\relax\sphinxstyleindexentry{EEMD\_Transfer\_Learning}\sphinxstyleindexpageref{EEMD:\detokenize{module-EEMD_Transfer_Learning}}
\indexspace
\bigletter{w}
\item\relax\sphinxstyleindexentry{WPT\_Feature\_Extraction}\sphinxstyleindexpageref{WPT:\detokenize{module-WPT_Feature_Extraction}}
\item\relax\sphinxstyleindexentry{WPT\_Transfer\_Learning}\sphinxstyleindexpageref{WPT:\detokenize{module-WPT_Transfer_Learning}}
\item\relax\sphinxstyleindexentry{WPT\_Transfer\_Learning\_2case}\sphinxstyleindexpageref{WPT:\detokenize{module-WPT_Transfer_Learning_2case}}
\end{sphinxtheindex}

\renewcommand{\indexname}{Index}
\printindex
\end{document}